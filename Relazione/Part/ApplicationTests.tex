\section{Application Tests}

\subsection{Unit Tests}

I seguenti test verificano il corretto funzionamento delle singole unità di codice.

\subsubsection{Test UT1: Registrazione di un FantaUser} \label{UT1}
\todo{Va tenuto qui?}

Questo unit test verifica il corretto funzionamento del sistema di registrazione di un FantaUser, 
tramite mock dei repository, implementando il template \hyperref[UC-01]{UC-01}.

\begin{lstlisting}[language=Java]
@Test
void testRegisterFantaUser_SavesWhenNotExists() {
    when(fantaUserRepository.getUser("mail", "pswd")).thenReturn(Optional.empty());

    service.registerFantaUser("mail", "pswd");

    verify(fantaUserRepository).saveFantaUser(argThat(user ->
            user.getEmail().equals("mail") && user.getPassword().equals("pswd")));
}
\end{lstlisting}

Similmente sono stati anche prodotti dei test per la registrazione di un NewsPaper.


\subsubsection{Test UT2: Login di un FantaUser} \label{UT2}
\todo{Va tenuto qui?}

Questo unit test verifica il corretto funzionamento del sistema di registrazione di un FantaUser, 
tramite mock dei repository.

\begin{lstlisting}[language=Java]
@Test
void testLoginFantaUser_ReturnsTrueWhenPresent() {
    when(fantaUserRepository.getUser("mail", "pswd"))
            .thenReturn(Optional.of(new FantaUser("mail", "pswd")));

    assertThat(service.loginFantaUser("mail", "pswd")).isTrue();
}
\end{lstlisting}

Similmente sono stati anche prodotti dei test per il login di un NewsPaper.


\subsubsection{Test UT3: Crea fantalega} \label{UT3}

Questo unit test verifica il corretto funzionamento del sistema di creazione di una nuova League,
tramite mock dei repository.

\begin{lstlisting}[language=Java]
@Test
void testCreateLeague() {
	FantaUser admin = new FantaUser("admin@test.com", "pwd");
	NewsPaper np = new NewsPaper("Gazzetta");
	String leagueCode = "L001";

	// League code does not exist yet
	when(leagueRepository.getLeagueByCode(leagueCode))
        .thenReturn(Optional.empty());

	adminUserService.createLeague("My League", admin, np, leagueCode);

	// Verify that saveLeague was called
	verify(leagueRepository, times(1)).saveLeague(any(League.class));
}
\end{lstlisting}


\subsubsection{Test UT4: Genera calendario} \label{UT4}

Questo unit test verifica il corretto funzionamento del sistema di generazione del calendario della League,
tramite mock dei repository.

\begin{lstlisting}[language=Java]
@Test
void testGenerateCalendar_SavesMatches() {
	FantaUser admin = new FantaUser(null, null);
	League league = new League(admin, "Serie A", null, null);

	// Create 4 real teams (even number for round-robin)
	FantaTeam team1 = new FantaTeam("Team1", null, 0, null, new HashSet<>());
	FantaTeam team2 = new FantaTeam("Team2", null, 0, null, new HashSet<>());
	FantaTeam team3 = new FantaTeam("Team3", null, 0, null, new HashSet<>());
	FantaTeam team4 = new FantaTeam("Team4", null, 0, null, new HashSet<>());
	List<FantaTeam> teams = List.of(team1, team2, team3, team4);

	// 38 match days (real objects are okay)
	List<MatchDaySerieA> matchDays = new ArrayList<>();
	for (int i = 0; i < 38; i++)
		matchDays.add(new MatchDaySerieA("match", LocalDate.now()));

	// Mock repositories
	when(fantaTeamRepository.getAllTeams(league)).thenReturn(teams);
	when(matchDayRepository.getAllMatchDays()).thenReturn(matchDays);

	adminUserService.generateCalendar(league);

    verify(matchRepository, atLeastOnce()).saveMatch(any(Match.class));
}
\end{lstlisting}


\subsubsection{Test UT5: Assegna giocatori alle rose} \label{UT5}

Questo unit test verifica il corretto funzionamento del sistema di assegnazione dei Player ai FantaTeam che li hanno acquistati,
tramite mock dei repository.

\begin{lstlisting}[language=Java]
@Test
void testSetPlayerToTeam_SavesContract_WhenBelowLimits() {
	FantaTeam team = new FantaTeam("Team", null, 0, null, new HashSet<>());
	Player.Goalkeeper gk = new Player.Goalkeeper("Gigi", "Buffon", Player.Club.JUVENTUS);

	adminUserService.setPlayerToTeam(team, gk);

	verify(contractRepository).saveContract(argThat(c -> c.getTeam().equals(team) && c.getPlayer().equals(gk)));
}
\end{lstlisting}


\subsubsection{Test UT6: Calcola voti della giornata} \label{UT6}

Questo unit test verifica il corretto funzionamento del sistema di calcolo dei voti della giornata una volta terminata,
tramite mock dei repository.

\begin{lstlisting}[language=Java]
@Test
void testCalculateGrades_SavesResultsAndUpdatesPoints() {

	FantaUser admin = new FantaUser("admin@example.com", "pwd");
	NewsPaper newspaper = new NewsPaper("Gazzetta");
    League league = new League(admin, "Serie A", newspaper, "1234");

	LocalDate matchDate = LocalDate.of(2025, 9, 21); // Sunday
	MatchDaySerieA prevDay = new MatchDaySerieA("Day0", matchDate.minusWeeks(1));
	MatchDaySerieA dayToCalc = new MatchDaySerieA("Day1", matchDate);

	when(matchDayRepository.getPreviousMatchDay(any()))
        .thenReturn(Optional.of(prevDay));

    AdminUserService serviceWithFixedDate = new AdminUserService(transactionManager) {
		@Override
		protected LocalDate today() {
			return matchDate.plusDays(5);
		}

		@Override
		protected Optional<MatchDaySerieA> getNextMatchDayToCalculate(LocalDate d, TransactionContext c, League l,
				FantaUser u) {
			return Optional.of(dayToCalc);
		}
	};

	// Teams
	FantaTeam team1 = new FantaTeam("Team1", league, 0, admin, Set.of());
	FantaTeam team2 = new FantaTeam("Team2", league, 0, admin, Set.of());

	// Match
	Match match = new Match(dayToCalc, team1, team2);
	when(matchRepository.getAllMatchesByMatchDay(dayToCalc, league))
		.thenReturn(List.of(new Match(dayToCalc, team1, team2)));

	// Players
	Player.Goalkeeper gk1 = new Player.Goalkeeper("G1", "Alpha", Player.Club.ATALANTA);
	Player.Goalkeeper gk2 = new Player.Goalkeeper("G2", "Beta", Player.Club.BOLOGNA);

	LineUp lineup1 = new _433LineUp._443LineUpBuilder(match, team1).withGoalkeeper(gk1).build();
	LineUp lineup2 = new _433LineUp._443LineUpBuilder(match, team2).withGoalkeeper(gk2).build();

	when(lineUpRepository.getLineUpByMatchAndTeam(match, team1)).thenReturn(Optional.of(lineup1));
	when(lineUpRepository.getLineUpByMatchAndTeam(match, team2)).thenReturn(Optional.of(lineup2));

	// Grades
	Grade grade1 = new Grade(gk1, dayToCalc, 70.0, newspaper);
	Grade grade2 = new Grade(gk2, dayToCalc, 60.0, newspaper);
	when(gradeRepository.getAllMatchGrades(match, newspaper)).thenReturn(List.of(grade1, grade2));

	// Act
	serviceWithFixedDate.calculateGrades(admin, league);

	// Assert: Result persisted
	verify(resultRepository).saveResult(any());

	// Assert: team points updated
	assertThat(team1.getPoints()).isEqualTo(3);
	assertThat(team2.getPoints()).isEqualTo(0);
}
\end{lstlisting}


\subsubsection{Test UT7: Unisciti alla lega} \label{UT7}

Questo unit test verifica il corretto funzionamento del sistema di ingresso in una League,
tramite mock dei repository.

\begin{lstlisting}[language=Java]
@Test
void testJoinLeague() {
	FantaUser user = new FantaUser("user@test.com", "pwd");
	League league = new League(user, "Test League", new NewsPaper("Gazzetta"), "L002");
	FantaTeam team = new FantaTeam("Team A", league, 0, user, Set.of());

	when(leagueRepository.getLeaguesByUser(user))
        .thenReturn(Collections.emptyList());

	userService.joinLeague(team, league);

	verify(teamRepository, times(1)).saveTeam(team);
}
\end{lstlisting}


\subsubsection{Test UT8: Visualizza calendario} \label{UT8}

Questo unit test verifica il corretto funzionamento del sistema di visualizzazione del calendario,
tramite mock dei repository.

\begin{lstlisting}[language=Java]
@Test
void testGetAllMatches() {
	League league = new League(null, null, null, null);
	MatchDaySerieA day1 = new MatchDaySerieA(null, null);
	Match m1 = new Match(day1, null, null);
	when(context.getMatchDayRepository().getAllMatchDays())
        .thenReturn(List.of(day1));
	when(context.getMatchRepository().getAllMatchesByMatchDay(day1, league))
        .thenReturn(List.of(m1));

	Map<MatchDaySerieA, List<Match>> result = userService.getAllMatches(league);
	assertThat(result.get(day1)).containsExactly(m1);
	}
\end{lstlisting}


\subsubsection{Test UT9: Inserisci formazione} \label{UT9}

Questo unit test verifica il corretto funzionamento del sistema di salvataggio della LineUp,
tramite mock dei repository.

\begin{lstlisting}[language=Java]
@Test
void testSaveLineUp() {
	FantaUser user = new FantaUser("user@test.com", "pwd");
	League league = new League(user, "Test League", new NewsPaper("Gazzetta"), "L003");
	MatchDaySerieA matchDay = new MatchDaySerieA("MD1", LocalDate.of(2025, 9, 15)); // Monday
	FantaTeam team = new FantaTeam("Dream Team", league, 30, user, new HashSet<>());
	Match match = new Match(matchDay, team, team);
	LineUp lineUp = new _433LineUp._443LineUpBuilder(match, team).build();

	UserService spyService = spy(userService);
	doReturn(team).when(spyService).getFantaTeamByUserAndLeague(league, user);
	doReturn(LocalDate.of(2025, 9, 15)).when(spyService).today(); // Current Monday

	// Stub repos
	when(context.getMatchDayRepository().getPreviousMatchDay(any()))
        .thenReturn(Optional.empty());
	when(context.getLineUpRepository().getLineUpByMatchAndTeam(match, team))
        .thenReturn(Optional.empty());

	spyService.saveLineUp(lineUp);

	verify(context.getLineUpRepository()).saveLineUp(lineUp);
}
\end{lstlisting}


\subsubsection{Test UT10: Visualizza prossimo Match} \label{UT10}

Questo unit test verifica il corretto funzionamento del sistema di visualizzazione del prossimo Match,
tramite mock dei repository.

\begin{lstlisting}[language=Java]
@Test
void testGetNextMatch() {
    League league = new League(null, null, null, null);
	FantaTeam team = new FantaTeam(null, league, 0, null, null);
	MatchDaySerieA prev = new MatchDaySerieA(null, null);
	MatchDaySerieA next = new MatchDaySerieA(null, null);
	Match prevMatch = new Match(next, team, team);
	Match nextMatch = new Match(next, team, team);

	when(context.getMatchDayRepository().getPreviousMatchDay(any()))
        .thenReturn(Optional.of(prev));
	when(context.getMatchRepository().getMatchByMatchDay(prev, league, team))
        .thenReturn(prevMatch);
	when(resultRepository.getResult(prevMatch))
        .thenReturn(Optional.of(mock(Result.class)));
	when(context.getMatchDayRepository().getNextMatchDay(any()))
        .thenReturn(Optional.of(next));
	when(context.getMatchRepository().getMatchByMatchDay(next, league, team))
        .thenReturn(nextMatch);

	Match result = userService.getNextMatch(league, team, LocalDate.now());
	assertThat(result).isEqualTo(nextMatch);
}
\end{lstlisting}


\subsubsection{Test UT11: Visualizza classifica} \label{UT11}

Questo unit test verifica il corretto funzionamento del sistema di visualizzazione della classifica,
tramite mock dei repository.

\begin{lstlisting}[language=Java]
@Test
void testGetStandings() {
	League league = new League(null, null, null, null);
	FantaTeam t1 = mock(FantaTeam.class);
	FantaTeam t2 = mock(FantaTeam.class);

	when(t1.getPoints()).thenReturn(10);
	when(t2.getPoints()).thenReturn(20);

	UserService spyService = spy(userService);
	doReturn(List.of(t1, t2)).when(spyService).getAllFantaTeams(league);

	List<FantaTeam> standings = spyService.getStandings(league);

	assertThat(standings).containsExactly(t2, t1);
}
\end{lstlisting}


\subsubsection{Test UT12: Scambia giocatori - Invia proposta} \label{UT12}

Questo unit test verifica il corretto funzionamento del sistema di invio di proposte di scambio di Player tra due FantaTeam
tramite mock dei repository.

\begin{lstlisting}[language=Java]
@Test
void testCreateProposal_HappyPath() {
	League league = new League(null, null, null, null);
	FantaUser user = new FantaUser(null, null);
	FantaTeam myTeam = spy(new FantaTeam("My Team", league, 0, user, new HashSet<>()));
	FantaTeam opponentTeam = new FantaTeam("Opponent", league, 0, user, new HashSet<>());
	Player offeredPlayer = new Player.Defender(null, null, null);
	Player requestedPlayer = new Player.Defender(null, null, null);

	Contract offeredContract = new Contract(myTeam, offeredPlayer);
	Contract requestedContract = new Contract(opponentTeam, requestedPlayer);
	myTeam.getContracts().add(offeredContract);
	opponentTeam.getContracts().add(requestedContract);

	when(context.getProposalRepository().getProposal(offeredContract, requestedContract))
		.thenReturn(Optional.empty());
	when(context.getProposalRepository().saveProposal(any()))
        .thenReturn(true);

	assertThat(userService.createProposal(requestedPlayer, offeredPlayer, myTeam, opponentTeam)).isTrue();
}
\end{lstlisting}


\subsubsection{Test UT13: Scambia giocatori - Accetta proposta} \label{UT13}

Questo unit test verifica il corretto funzionamento del sistema di accettazione di una proposta di scambio di Player 
tramite mock dei repository.

\begin{lstlisting}[language=Java]
@Test
void testAcceptProposal() {
	// Setup teams and players
	FantaTeam myTeam = mock(FantaTeam.class);
	FantaTeam offeringTeam = new FantaTeam(null, null, 0, null, null);
	Player offeredPlayer = new Player.Forward(null, null, null);
	Player requestedPlayer = new Player.Midfielder(null, null, null);

	// Contracts
	Contract offeredContract = mock(Contract.class);
	when(offeredContract.getTeam()).thenReturn(offeringTeam);
	when(offeredContract.getPlayer()).thenReturn(offeredPlayer);

    Contract requestedContract = mock(Contract.class);
    when(requestedContract.getTeam()).thenReturn(myTeam);
	when(requestedContract.getPlayer()).thenReturn(requestedPlayer);

	// Proposal
	Proposal.PendingProposal proposal = mock(Proposal.PendingProposal.class);
	when(proposal.getRequestedContract()).thenReturn(requestedContract);
	when(proposal.getOfferedContract()).thenReturn(offeredContract);

	// Stub isSameTeam
	when(myTeam.isSameTeam(myTeam)).thenReturn(true);

	// Stub searchContract to return non-empty Optionals
	UserService userServiceSpy = spy(userService);
	doReturn(Optional.of(requestedContract))
        .when(userServiceSpy).searchContract(myTeam, requestedPlayer);
	doReturn(Optional.of(offeredContract))
        .when(userServiceSpy).searchContract(offeringTeam, offeredPlayer);

	// Run test
	userServiceSpy.acceptProposal(proposal, myTeam);

	// Verify repository interactions
	verify(context.getContractRepository())
        .deleteContract(requestedContract);
	verify(context.getContractRepository())
        .deleteContract(offeredContract);
	verify(context.getContractRepository(), times(2)).saveContract(any(Contract.class));
	verify(context.getProposalRepository()).deleteProposal(proposal);
}
\end{lstlisting}


\subsubsection{Test UT14: Visualizza squadre} \label{UT14}

Questo unit test verifica il corretto funzionamento del sistema di visualizzazione di tutti i FantaTeam 
tramite mock dei repository.

\begin{lstlisting}[language=Java]
@Test
void testGetAllFantaTeams() {
    League league = new League(null, "My League", new NewsPaper("Gazzetta"), "L999");
	FantaTeam t1 = new FantaTeam("Team 1", league, 0, new FantaUser("u1", "pwd"), Set.of());
	FantaTeam t2 = new FantaTeam("Team 2", league, 0, new FantaUser("u2", "pwd"), Set.of());

	when(context.getTeamRepository().getAllTeams(league))
        .thenReturn(List.of(t1, t2));

	List<FantaTeam> result = userService.getAllFantaTeams(league);

	assertThat(result).containsExactly(t1, t2);
	verify(context.getTeamRepository(), times(1)).getAllTeams(league);
}
\end{lstlisting}


\subsubsection{Test UT15: Visualizza listone giocatori} \label{UT15}

Questo unit test verifica il corretto funzionamento del sistema di visualizzazione di tutti i giocatori 
tramite mock dei repository.

\begin{lstlisting}[language=Java]
@Test
void testGetAllPlayers() {
	Player p1 = new Player.Goalkeeper("", "", null);
	Player p2 = new Player.Goalkeeper("", "", null);
	when(context.getPlayerRepository().findAll()).thenReturn(List.of(p1, p2));

	List<Player> result = userService.getAllPlayers();
	assertThat(result).containsExactly(p1, p2);
}
\end{lstlisting}


\subsubsection{Test UT16: Assegna voti ai giocatori} \label{UT16}

Questo unit test verifica il corretto funzionamento del sistema di assegnazione dei voti ai Player tramite mock dei repository.

\begin{lstlisting}[language=Java]
@Test
void testSetVoteToPlayers_MultipleGrades() {
	Grade grade2 = mock(Grade.class);
	when(grade2.getMatchDay()).thenReturn(matchDay);
	when(grade2.getMark()).thenReturn(15.0);

	NewsPaperService spyService = spy(service);
	doReturn(Optional.of(matchDay)).when(spyService).getMatchDay();

	spyService.setVoteToPlayers(Set.of(grade, grade2));

	verify(gradeRepository).saveGrade(grade);
	verify(gradeRepository).saveGrade(grade2);
}
\end{lstlisting}


\subsubsection{Test relativi ai JpaRepositories}

Per ogni repository implementata tramite JPA è stata realizzata una classe di test per verificare la 
corretta interazione con il database, sia per salvare nuovi elementi, sia per recuperare dati. 
Di seguito è proposto un test che verifica il corretto salvataggio nel database di un nuovo FantaTeam.

\begin{lstlisting}[language=Java]
@Test
@DisplayName("saveTeam() should persist correctly")
public void testSaveTeam() {

	entityManager.getTransaction().begin();

	FantaUser user = new FantaUser("mail1", "pswd1");
	FantaTeam team = new FantaTeam("team1", league, 0, user, new HashSet<Contract>());

	entityManager.persist(user);

	fantaTeamRepository.saveTeam(team);

	FantaTeam result = entityManager
		.createQuery("FROM FantaTeam t WHERE t.league = :league AND t.fantaManager = :user", FantaTeam.class)
		.setParameter("league", league).setParameter("user", user).getSingleResult();

	assertThat(result).isEqualTo(team);

    entityManager.close();
}
\end{lstlisting}

Di seguito è invece proposto un test che verifica che il recupero di informazioni dal database sia corretto: 
in particolare, l'obiettivo è recuperare tutti i voti assegnati in un singolo match.

\begin{lstlisting}[language=Java]
@Test
@DisplayName("getAllMatchGrades() when two grades have been persisted")
public void testGetAllMatchGradesWhenTwoGradesExist() {
	Player player1 = new Player.Goalkeeper("Gigi", "Buffon", Club.JUVENTUS);
	Player player2 = new Player.Forward("Gigi", "Riva", Club.CAGLIARI);

    Grade voto1 = new Grade(player1, matchDay, 6.0, newsPaper);
	Grade voto2 = new Grade(player1, matchDay, 8.0, newsPaper);
	Contract contract1 = new Contract(team1, player1);
	Contract contract2 = new Contract(team1, player2);
	sessionFactory.inTransaction(session -> {
		session.persist(player1);
		session.persist(player2);
		session.persist(voto1);
		session.persist(voto2);
		session.persist(contract1);
		session.persist(contract2);
	});

	assertThat(gradeRepository.getAllMatchGrades(match, newsPaper)).containsExactly(voto1, voto2);
}
\end{lstlisting}


\subsubsection{Test relativi al Domain Model}

Inoltre, sono stati creati degli unit test per verificare il corretto comportamento di alcuni metodi presenti nelle classi del Domain Model:
\begin{itemize}
    \item la classe \textit{\_433LineUp} usa il pattern \textbf{Builder}, perciò si testa che la costruzione di nuovi oggetti sia corretta;
    \item la classe \textit{FantaTeamViewer}, sfruttando il pattern \textbf{Visitor}, si occupa di recuperare gli 
        insiemi di giocatori appartenenti ad un FantaTeam divisi per ruoli basandosi sui contratti del FantaTeam;
    \item la classe \textit{LineUpViewer}, similmente a \textit{FantaTeamViewer}, si occupa di recuperare gli 
        insiemi di giocatori presenti in una LineUp divisi per ruoli.
\end{itemize}



\subsection{Integration Tests}

I seguenti test verificano che le varie componenti cooperino correttamente per l'implementazione degli use case richiesti.


\subsubsection{Test IT1: Crea fantalega} \label{IT1}

Questo unit test verifica il corretto funzionamento del sistema di creazione di una nuova League,
implementando il template \hyperref[UC-01]{UC-01}.

\begin{lstlisting}[language=Java]
@Test
void createLeague() {

	entityManager.getTransaction().begin();

	FantaUser admin = new FantaUser("mail", "pswd");
	fantaUserRepository.saveFantaUser(admin);

	NewsPaper newsPaper = new NewsPaper("Gazzetta");
	newspaperRepository.saveNewsPaper(newsPaper);

	entityManager.getTransaction().commit();

	adminUserService.createLeague("lega", admin, newsPaper, "1234");

    Optional<League> result = leagueRepository.getLeagueByCode("1234");

	assertThat(result.isPresent());
	League league = result.get();

	assertThat(league.getName()).isEqualTo("lega");
	assertThat(league.getAdmin()).isEqualTo(admin);
	assertThat(league.getNewsPaper()).isEqualTo(newsPaper);
	assertThat(league.getLeagueCode()).isEqualTo("1234");
}
\end{lstlisting}

\begin{comment}
\subsubsection{Test IT4: Genera calendario} \label{IT4}

Questo unit test verifica il corretto funzionamento del sistema di generazione del calendario della League,
implementando il template XXXXXXXX \todo{collegare al template}.

\begin{lstlisting}[language=Java]
@Test
void generateCalendar() {

	entityManager.getTransaction().begin();

	FantaUser admin = new FantaUser("mail", "pswd");
	FantaUser user = new FantaUser("user2", "pswd2");
	fantaUserRepository.saveFantaUser(admin);
	fantaUserRepository.saveFantaUser(user);

	NewsPaper newsPaper = new NewsPaper("Gazzetta");
	newspaperRepository.saveNewsPaper(newsPaper);

	League league = new League(admin, "lega", newsPaper, "0000");
	leagueRepository.saveLeague(league);

	FantaTeam team1 = new FantaTeam("team1", league, 0, admin, new HashSet<Contract>());
	FantaTeam team2 = new FantaTeam("team2", league, 0, user, new HashSet<Contract>());

	fantaTeamRepository.saveTeam(team1);
	fantaTeamRepository.saveTeam(team2);

	List<MatchDaySerieA> matchDays = new ArrayList<MatchDaySerieA>();
	for (int i = 0; i < 38; i++) {
		matchDays.add(new MatchDaySerieA("Match " + String.valueOf(i), LocalDate.of(2025, 9, 7).plusWeeks(i)));
	}

    sessionFactory.inTransaction(t -> {
		for (MatchDaySerieA matchDaySerieA : matchDays) {
			t.persist(matchDaySerieA);
		}
	});

	entityManager.getTransaction().commit();

    adminUserService.generateCalendar(league);

	for (MatchDaySerieA matchDaySerieA : matchDays) {
		Match matchByMatchDay = matchRepository.getMatchByMatchDay(matchDaySerieA, league, team1);

		assertThat(matchByMatchDay.getMatchDaySerieA()).isEqualTo(matchDaySerieA);
		assertThat(matchByMatchDay.getTeam1().equals(team1) || matchByMatchDay.getTeam2().equals(team1)).isTrue();
	}
}
\end{lstlisting}

\end{comment}

\subsubsection{Test IT2: Assegna giocatori alle rose} \label{IT2}

Questo unit test verifica il corretto funzionamento del sistema di assegnazione dei Player ai FantaTeam che li hanno acquistati,
implementando il template \hyperref[UC-02]{UC-02}.

\begin{lstlisting}[language=Java]
@Test
void setPlayersToTeam() {

	entityManager.getTransaction().begin();

    FantaUser admin = new FantaUser("mail", "pswd");
	fantaUserRepository.saveFantaUser(admin);

    NewsPaper newsPaper = new NewsPaper("Gazzetta");
	newspaperRepository.saveNewsPaper(newsPaper);

    League league = new League(admin, "lega", newsPaper, "1234");
	leagueRepository.saveLeague(league);

	FantaTeam team = new FantaTeam("", league, 0, admin, Set.of());
	fantaTeamRepository.saveTeam(team);

	Player.Forward player = new Player.Forward("Lionel", "Messi", Club.CREMONESE);
	Player.Goalkeeper player2 = new Player.Goalkeeper("Gigi", "Buffon", Club.JUVENTUS);
	playerRepository.addPlayer(player);
	playerRepository.addPlayer(player2);

	entityManager.getTransaction().commit();

	adminUserService.setPlayerToTeam(team, player);
	adminUserService.setPlayerToTeam(team, player2);

	Optional<Contract> contract = contractRepository.getContract(team, player);
	assertThat(contract).isPresent();
	Contract found = contract.get();
	assertThat(found.getPlayer()).isEqualTo(player);
	assertThat(found.getTeam()).isEqualTo(team);

    Optional<Contract> contract2 = contractRepository.getContract(team, player);
	assertThat(contract2).isPresent();
	Contract found2 = contract.get();
	assertThat(found2.getPlayer()).isEqualTo(player);
	assertThat(found2.getTeam()).isEqualTo(team);

}
\end{lstlisting}


\begin{comment}
\subsubsection{Test IT6: Calcola voti della giornata} \label{IT6}

Questo unit test verifica il corretto funzionamento del sistema di calcolo dei voti della giornata una volta terminata,
implementando il template XXXXXXXX \todo{collegare al template}.

\begin{lstlisting}[language=Java]
@Test
void calculateGrades() {

	entityManager.getTransaction().begin();

	// Users
	FantaUser admin = new FantaUser("mail", "pswd");
	FantaUser user = new FantaUser("mail2", "pswd2");

	fantaUserRepository.saveFantaUser(admin);
	fantaUserRepository.saveFantaUser(user);

	NewsPaper newsPaper = new NewsPaper("Gazzetta");
	newspaperRepository.saveNewsPaper(newsPaper);

	League league = new League(admin, "lega", newsPaper, "0000");
	leagueRepository.saveLeague(league);

	// Teams
	FantaTeam team1 = new FantaTeam("team1", league, 0, admin, new HashSet<Contract>());
	FantaTeam team2 = new FantaTeam("team2", league, 0, user, new HashSet<Contract>());

	fantaTeamRepository.saveTeam(team1);
	fantaTeamRepository.saveTeam(team2);

	// MatchDays
	LocalDate matchDate = LocalDate.of(2025, 9, 14);
	MatchDaySerieA prevDay = new MatchDaySerieA("Day0", matchDate.minusWeeks(1));
	MatchDaySerieA dayToCalc = new MatchDaySerieA("Day1", matchDate);

	entityManager.persist(prevDay);
	entityManager.persist(dayToCalc);

	// Match
	Match prevMatch = new Match(prevDay, team1, team2);
	Match match = new Match(dayToCalc, team1, team2);

	matchRepository.saveMatch(prevMatch);
	matchRepository.saveMatch(match);

	// Players
	Goalkeeper gk1 = new Goalkeeper("Gianluigi", "Buffon", Club.JUVENTUS);
	Goalkeeper gk2 = new Goalkeeper("Samir", "Handanovic", Club.INTER);

	Defender d1 = new Defender("Paolo", "Maldini", Club.MILAN);
	Defender d2 = new Defender("Franco", "Baresi", Club.JUVENTUS);
	Defender d3 = new Defender("Alessandro", "Nesta", Club.LAZIO);
	Defender d4 = new Defender("Giorgio", "Chiellini", Club.JUVENTUS);
	Defender d5 = new Defender("Leonardo", "Bonucci", Club.JUVENTUS);

	Midfielder m1 = new Midfielder("Andrea", "Pirlo", Club.JUVENTUS);
    Midfielder m2 = new Midfielder("Daniele", "De Rossi", Club.ROMA);
	Midfielder m3 = new Midfielder("Marco", "Verratti", Club.CREMONESE);
	Midfielder m4 = new Midfielder("Claudio", "Marchisio", Club.JUVENTUS);

	Forward f1 = new Forward("Roberto", "Baggio", Club.BOLOGNA);
	Forward f2 = new Forward("Francesco", "Totti", Club.ROMA);
	Forward f3 = new Forward("Alessandro", "Del Piero", Club.JUVENTUS);
	Forward f4 = new Forward("Lorenzo", "Insigne", Club.NAPOLI);

	List<Player> players = List.of(gk1, gk2, d1, d2, d3, d4, d5, m1, m2, m3, m4, f1, f2, f3, f4);

	players.forEach(playerRepository::addPlayer);

	for (Player player : players) {
		contractRepository.saveContract(new Contract(team1, player));
		contractRepository.saveContract(new Contract(team2, player));
	}

	// LineUps
	LineUp lineup1 = new _433LineUp._443LineUpBuilder(match, team1).withGoalkeeper(gk1)
			.withDefenders(d1, d2, d3, d4).withMidfielders(m1, m2, m3).withForwards(f1, f2, f3)
			.withSubstituteGoalkeepers(List.of(gk2)).withSubstituteDefenders(List.of(d5))
			.withSubstituteMidfielders(List.of(m4)).withSubstituteForwards(List.of(f4)).build();
	LineUp lineup2 = new _433LineUp._443LineUpBuilder(match, team2).withGoalkeeper(gk1)
			.withDefenders(d1, d2, d3, d4).withMidfielders(m1, m2, m3).withForwards(f1, f2, f3)
			.withSubstituteGoalkeepers(List.of(gk2)).withSubstituteDefenders(List.of(d5))
			.withSubstituteMidfielders(List.of(m4)).withSubstituteForwards(List.of(f4)).build();

	lineUpRepository.saveLineUp(lineup1);
	lineUpRepository.saveLineUp(lineup2);

	// Grades
	Grade grade1 = new Grade(gk1, dayToCalc, 70.0, newsPaper);
	Grade grade2 = new Grade(gk2, dayToCalc, 60.0, newsPaper);

	gradeRepository.saveGrade(grade1);
	gradeRepository.saveGrade(grade2);

	entityManager.getTransaction().commit();

	AdminUserService service = new AdminUserService(transactionManager) {
		@Override
		protected LocalDate today() {
			return LocalDate.of(2025, 9, 16); // after 14/09
		}
	};

	service.calculateGrades(admin, league);

	List<Grade> allMatchGrades = gradeRepository.getAllMatchGrades(match, newsPaper);
	assertThat(allMatchGrades.size()).isEqualTo(2);
	assertThat(allMatchGrades.get(0)).isEqualTo(grade1);
	assertThat(allMatchGrades.get(1)).isEqualTo(grade2);

}
\end{lstlisting}


\subsubsection{Test IT7: Unisciti alla lega} \label{IT7}

Questo unit test verifica il corretto funzionamento del sistema di ingresso in una League,
implementando il template XXXXXXXX \todo{collegare al template}.

\begin{lstlisting}[language=Java]
@Test
void joinLeague() {

	entityManager.getTransaction().begin();
	
	FantaUser user = new FantaUser("user@test.com", "pwd");
	fantaUserRepository.saveFantaUser(user);
		
	NewsPaper newsPaper = new NewsPaper("Gazzetta");
	newspaperRepository.saveNewsPaper(newsPaper);
		
	League league = new League(user, "Test League", newsPaper, "1234");
	leagueRepository.saveLeague(league);
		
	FantaTeam team = new FantaTeam("Team A", league, 0, user, new HashSet<Contract>());

	entityManager.getTransaction().commit();
		
	userService.joinLeague(team, league);
		
	Optional<League> result = leagueRepository.getLeagueByCode("1234");
	assertThat(result).isPresent();
		
	League resultLeague = result.get();
	assertThat(resultLeague.getAdmin()).isEqualTo(user);
	assertThat(resultLeague.getLeagueCode()).isEqualTo("1234");
	assertThat(resultLeague.getName()).isEqualTo("Test League");
	assertThat(resultLeague.getNewsPaper()).isEqualTo(newsPaper);
}
\end{lstlisting}


\subsubsection{Test IT8: Visualizza calendario} \label{IT8}

Questo unit test verifica il corretto funzionamento del sistema di visualizzazione del calendario, cioè della lista dei Match,
implementando il template XXXXXXXX \todo{collegare al template}.

\begin{lstlisting}[language=Java]
@Test
void testGetAllMatches() {
		
	entityManager.getTransaction().begin();
		
	FantaUser user1 = new FantaUser("user1@test.com", "pwd");
	fantaUserRepository.saveFantaUser(user1);
	FantaUser user2 = new FantaUser("user2@test.com", "pwd");
	fantaUserRepository.saveFantaUser(user2);
		
	NewsPaper newsPaper = new NewsPaper("Gazzetta");
	newspaperRepository.saveNewsPaper(newsPaper);
		
	League league = new League(user1, "Test League", newsPaper, "1234");
	leagueRepository.saveLeague(league);
		
	FantaTeam team1 = new FantaTeam("Team A", league, 0, user1, new HashSet<Contract>());
	fantaTeamRepository.saveTeam(team1);
	FantaTeam team2 = new FantaTeam("Team B", league, 0, user2, new HashSet<Contract>());
	fantaTeamRepository.saveTeam(team2);
		
	MatchDaySerieA day1 = new MatchDaySerieA("MD1", LocalDate.of(2025, 9, 7));
	matchDayRepository.saveMatchDay(day1);
		
	Match m1 = new Match(day1, team1, team2);
	matchRepository.saveMatch(m1);
		
	entityManager.getTransaction().commit();
		
	Map<MatchDaySerieA, List<Match>> result = userService.getAllMatches(league);		
		
	assertThat(result.get(day1).size()).isEqualTo(1);
	Match resultMatch = result.get(day1).get(0);
		
	assertThat(resultMatch.getMatchDaySerieA().getName()).isEqualTo("MD1");
}
\end{lstlisting}

\end{comment}

\subsubsection{Test IT3: Inserisci formazione} \label{IT3}

Questo unit test verifica il corretto funzionamento del sistema di salvataggio della LineUp,
implementando il template \hyperref[UC-03]{UC-03}.

\begin{lstlisting}[language=Java]
@Test
void testSaveLineUp() {
		
	entityManager.getTransaction().begin();
		
	FantaUser user = new FantaUser("user@test.com", "pwd");
	fantaUserRepository.saveFantaUser(user);
		
	NewsPaper newsPaper = new NewsPaper("Gazzetta");
	newspaperRepository.saveNewsPaper(newsPaper);
		
	League league = new League(user, "Test League", newsPaper, "L003");
	leagueRepository.saveLeague(league);
		
	MatchDaySerieA matchDay = new MatchDaySerieA("MD1", LocalDate.now().plusWeeks(1)); // Monday
	matchDayRepository.saveMatchDay(matchDay);
		
	FantaTeam team = new FantaTeam("Dream Team", league, 30, user, new HashSet<>());
	fantaTeamRepository.saveTeam(team);
		
	Match match = new Match(matchDay, team, team);
	matchRepository.saveMatch(match);
		
	LineUp lineUp = new _433LineUp._443LineUpBuilder(match, team).build();
		
	entityManager.getTransaction().commit();
		
	userService.saveLineUp(lineUp);
		
	Optional<LineUp> result = lineUpRepository.getLineUpByMatchAndTeam(match, team);
	assertThat(result).isPresent();
		
	LineUp resultLineUp = result.get();
	assertThat(resultLineUp.getMatch()).isEqualTo(match);
	assertThat(resultLineUp.getTeam()).isEqualTo(team);
		
}
\end{lstlisting}

\begin{comment}

\subsubsection{Test IT10: Visualizza prossimo Match} \label{IT10}

Questo unit test verifica il corretto funzionamento del sistema di visualizzazione del prossimo Match,
implementando il template XXXXXXXX \todo{collegare al template}.

\begin{lstlisting}[language=Java]
@Test
void testGetNextMatch() {

	entityManager.getTransaction().begin();

	FantaUser user = new FantaUser("user@test.com", "pwd");
	fantaUserRepository.saveFantaUser(user);

	NewsPaper newsPaper = new NewsPaper("Gazzetta");
	newspaperRepository.saveNewsPaper(newsPaper);

	League league = new League(user, "Test League", newsPaper, "L003");
	leagueRepository.saveLeague(league);

	FantaTeam team = new FantaTeam("Team", league, 30, user, new HashSet<>());
	FantaTeam team2 = new FantaTeam("Team2", league, 40, user, new HashSet<>());
	fantaTeamRepository.saveTeam(team);
	fantaTeamRepository.saveTeam(team2);

	MatchDaySerieA prevMatchDay = new MatchDaySerieA("MD1", LocalDate.now().minusWeeks(1));
	MatchDaySerieA nextMatchDay = new MatchDaySerieA("MD2", LocalDate.now().plusWeeks(1));
	matchDayRepository.saveMatchDay(prevMatchDay);
	matchDayRepository.saveMatchDay(nextMatchDay);

	Match prevMatch = new Match(prevMatchDay, team, team2);
	Match nextMatch = new Match(nextMatchDay, team, team2);
	matchRepository.saveMatch(prevMatch);
	matchRepository.saveMatch(nextMatch);

	Result prevResults = new Result(20, 50, 1, 2, prevMatch);
	resultsRepository.saveResult(prevResults);

	entityManager.getTransaction().commit();

	Match result = userService.getNextMatch(league, team, LocalDate.now());
	assertThat(result.getMatchDaySerieA().getName()).isEqualTo("MD2");
}
\end{lstlisting}


\subsubsection{Test IT11: Visualizza classifica} \label{IT11}

Questo unit test verifica il corretto funzionamento del sistema di visualizzazione della classifica,
implementando il template XXXXXXXX \todo{collegare al template}.

\begin{lstlisting}[language=Java]
@Test
void testGetStandings() {

	entityManager.getTransaction().begin();
		
	FantaUser user = new FantaUser("user@test.com", "pwd");
	fantaUserRepository.saveFantaUser(user);

	NewsPaper newsPaper = new NewsPaper("Gazzetta");
	newspaperRepository.saveNewsPaper(newsPaper);

	League league = new League(user, "Test League", newsPaper, "L003");
	leagueRepository.saveLeague(league);

	FantaTeam team1 = new FantaTeam("Team1", league, 10, user, new HashSet<>());
	FantaTeam team2 = new FantaTeam("Team2", league, 70, user, new HashSet<>());
	fantaTeamRepository.saveTeam(team1);
	fantaTeamRepository.saveTeam(team2);
		
	entityManager.getTransaction().commit();

	List<FantaTeam> standings = userService.getStandings(league);

	assertThat(standings.get(0).getName()).isEqualTo("Team2");
	assertThat(standings.get(1).getName()).isEqualTo("Team1");
}
\end{lstlisting}

\end{comment}

\subsubsection{Test IT4: Scambia giocatori - Invia proposta} \label{IT4}

Questo unit test verifica il corretto funzionamento del sistema di invio di proposte di scambio di Player tra due FantaTeam,
implementando il template \hyperref[UC-04]{UC-04}.

\begin{lstlisting}[language=Java]
@Test
void testCreateProposal_HappyPath() {

	entityManager.getTransaction().begin();

	FantaUser user = new FantaUser("user@test.com", "pwd");
	fantaUserRepository.saveFantaUser(user);

	NewsPaper newsPaper = new NewsPaper("Gazzetta");
	newspaperRepository.saveNewsPaper(newsPaper);

	League league = new League(user, "Test League", newsPaper, "L003");
	leagueRepository.saveLeague(league);

    FantaTeam myTeam = new FantaTeam("My Team", league, 0, user, new HashSet<>());
    FantaTeam opponentTeam = new FantaTeam("Opponent", league, 0, user, new HashSet<>());
	fantaTeamRepository.saveTeam(myTeam);
	fantaTeamRepository.saveTeam(opponentTeam);
		
	Player offeredPlayer = new Player.Defender("Mario", "Rossi", Club.ATALANTA);
	Player requestedPlayer = new Player.Defender("Luigi", "Verdi", Club.BOLOGNA);
	playerRepository.addPlayer(offeredPlayer);
	playerRepository.addPlayer(requestedPlayer);

	Contract offeredContract = new Contract(myTeam, offeredPlayer);
	Contract requestedContract = new Contract(opponentTeam, requestedPlayer);
	myTeam.getContracts().add(offeredContract);
	opponentTeam.getContracts().add(requestedContract);
	contractRepository.saveContract(offeredContract);
	contractRepository.saveContract(requestedContract);
		
	entityManager.getTransaction().commit();

	assertThat(userService.createProposal(requestedPlayer, offeredPlayer, myTeam, opponentTeam)).isTrue();
		
	Optional<Proposal> result = proposalRepository.getProposal(offeredContract, requestedContract);
	assertThat(result).isPresent();
	Proposal resultProposal = result.get();
	assertThat(resultProposal.getOfferedContract()).isEqualTo(offeredContract);
	assertThat(resultProposal.getRequestedContract()).isEqualTo(requestedContract);
}
\end{lstlisting}


\subsubsection{Test IT5: Scambia giocatori - Accetta proposta} \label{IT5}

Questo unit test verifica il corretto funzionamento del sistema di accettazione di una proposta di scambio di Player,
implementando il template \hyperref[UC-05]{UC-05}.

\begin{lstlisting}[language=Java]
@Test
void testAcceptProposal() {

	entityManager.getTransaction().begin();

	FantaUser user = new FantaUser("user@test.com", "pwd");
	fantaUserRepository.saveFantaUser(user);

	NewsPaper newsPaper = new NewsPaper("Gazzetta");
	newspaperRepository.saveNewsPaper(newsPaper);

	League league = new League(user, "Test League", newsPaper, "L003");
	leagueRepository.saveLeague(league);

	FantaTeam myTeam = new FantaTeam("My Team", league, 0, user, new HashSet<>());
	FantaTeam offeringTeam = new FantaTeam("Opponent", league, 0, user, new HashSet<>());
	fantaTeamRepository.saveTeam(myTeam);
	fantaTeamRepository.saveTeam(offeringTeam);
		
	Player requestedPlayer = new Player.Defender("Luigi", "Verdi", Club.BOLOGNA);
	Player offeredPlayer = new Player.Defender("Mario", "Rossi", Club.ATALANTA);
	playerRepository.addPlayer(requestedPlayer);
	playerRepository.addPlayer(offeredPlayer);

	Contract offeredContract = new Contract(offeringTeam, offeredPlayer);
	Contract requestedContract = new Contract(myTeam, requestedPlayer);
	offeringTeam.getContracts().add(offeredContract);
	myTeam.getContracts().add(requestedContract);
	contractRepository.saveContract(offeredContract);
	contractRepository.saveContract(requestedContract);
		
	Proposal.PendingProposal proposal = new PendingProposal(offeredContract, requestedContract);
	proposalRepository.saveProposal(proposal);
		
	entityManager.getTransaction().commit();

	userService.acceptProposal(proposal, myTeam);
		
	assertThat(contractRepository.getContract(offeringTeam, offeredPlayer)).isEmpty();
	assertThat(contractRepository.getContract(offeringTeam, requestedPlayer)).isPresent();
		
	assertThat(contractRepository.getContract(myTeam, requestedPlayer)).isEmpty();
	assertThat(contractRepository.getContract(myTeam, offeredPlayer)).isPresent();
}
\end{lstlisting}

\begin{comment}

\subsubsection{Test IT14: Visualizza squadre} \label{IT14}

Questo unit test verifica il corretto funzionamento del sistema di visualizzazione di tutti i FantaTeam,
implementando il template XXXXXXXX \todo{collegare al template}.

\begin{lstlisting}[language=Java]
@Test
void testGetAllFantaTeams() {

	entityManager.getTransaction().begin();

	FantaUser user = new FantaUser("user@test.com", "pwd");
	fantaUserRepository.saveFantaUser(user);

	NewsPaper newsPaper = new NewsPaper("Gazzetta");
	newspaperRepository.saveNewsPaper(newsPaper);

	League league = new League(user, "Test League", newsPaper, "L003");
	leagueRepository.saveLeague(league);

	FantaTeam team1 = new FantaTeam("Team1", league, 10, user, new HashSet<>());
	FantaTeam team2 = new FantaTeam("Team2", league, 70, user, new HashSet<>());
	fantaTeamRepository.saveTeam(team1);
	fantaTeamRepository.saveTeam(team2);

	entityManager.getTransaction().commit();
		
	List<FantaTeam> result = userService.getAllFantaTeams(league);

	assertThat(result.size()).isEqualTo(2);
	assertThat(result.get(0).getLeague()).isEqualTo(result.get(1).getLeague());
	assertThat(result.get(0).getName()).isIn("Team1", "Team2");
	assertThat(result.get(1).getName()).isIn("Team1", "Team2");
	assertThat(result.get(0).getPoints()).isIn(10, 70);
	assertThat(result.get(1).getPoints()).isIn(10, 70);
}
\end{lstlisting}


\subsubsection{Test IT15: Visualizza listone giocatori} \label{IT15}

Questo unit test verifica il corretto funzionamento del sistema di visualizzazione di tutti i giocatori,
implementando il template XXXXXXXX \todo{collegare al template}.

\begin{lstlisting}[language=Java]
@Test
void testGetAllPlayers() {

	entityManager.getTransaction().begin();

	Player p1 = new Player.Defender("Mario", "Rossi", Club.ATALANTA);
	Player p2 = new Player.Defender("Luigi", "Verdi", Club.BOLOGNA);
	playerRepository.addPlayer(p1);
	playerRepository.addPlayer(p2);

	entityManager.getTransaction().commit();

	List<Player> result = userService.getAllPlayers();
	assertThat(result).containsExactly(p1, p2);
}
\end{lstlisting}


\subsubsection{Test IT16: Assegna voti ai giocatori} \label{IT16}

Questo unit test verifica il corretto funzionamento del sistema di assegnazione dei voti ai Player,
implementando il template XXXXXXXX \todo{collegare al template}.

\begin{lstlisting}[language=Java]
@Test
public void assignGradesToPlayers() {

	entityManager.getTransaction().begin();

    FantaUser admin = new FantaUser("mail", "pswd");
	fantaUserRepository.saveFantaUser(admin);

	NewsPaper newsPaper = new NewsPaper("Gazzetta");
	newspaperRepository.saveNewsPaper(newsPaper);

	Player player = new Player.Forward("player", "1", Club.ATALANTA);
	Player player2 = new Player.Forward("player", "2", Club.ATALANTA);
	playerRepository.addPlayer(player);
	playerRepository.addPlayer(player2);

	MatchDaySerieA previousDay = new MatchDaySerieA("prima giornata", LocalDate.of(2020, 1, 13));
	MatchDaySerieA matchDay = new MatchDaySerieA("seconda giornata", LocalDate.of(2025, 9, 20));
	MatchDaySerieA nextDay = new MatchDaySerieA("terza giornata", LocalDate.of(2020, 1, 26));
	entityManager.persist(previousDay);
	entityManager.persist(matchDay);
	entityManager.persist(nextDay);

	League league = new League(admin, "lega", newsPaper, "1234");
	leagueRepository.saveLeague(league);

	FantaTeam team1 = new FantaTeam("", league, 0, admin, Set.of());
	FantaTeam team2 = new FantaTeam("", league, 0, admin, Set.of());
	fantaTeamRepository.saveTeam(team1);
	fantaTeamRepository.saveTeam(team2);

	Match match = new Match(matchDay, team1, team2);
	matchRepository.saveMatch(match);

	Grade grade = new Grade(player, matchDay, 25.0, newsPaper);
	Grade grade2 = new Grade(player2, matchDay, 20.0, newsPaper);

	entityManager.getTransaction().commit();

    newspaperservice = new NewsPaperService(transactionManager) {
		@Override
		protected LocalDate today() {
			return LocalDate.of(2025, 9, 22); // after 21/09
		}
	};

	Set<Player> players = newspaperservice.getPlayersToGrade(Player.Club.ATALANTA);
	assertThat(players.size()).isEqualTo(2);
	assertThat(players).anyMatch(t -> t.getName() == "player" && t.getSurname() == "1");
	assertThat(players).anyMatch(t -> t.getName() == "player" && t.getSurname() == "2");

	newspaperservice.setVoteToPlayers(Set.of(grade, grade2));
	assertThat(gradeRepository.getAllMatchGrades(match, newsPaper));

}
\end{lstlisting}

\end{comment}