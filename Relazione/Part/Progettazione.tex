\section{Progettazione}
\todo{Aggiungere UML, UseCases diagram e template, MockUps, Navigation Diagram ed ER? meglio dire come gestiamo il database in memoria e jpa/hibernate}
\todo[color=blue]{Andre scrivi il paragrafo 3 sul database ovvero parla di hibernate,transaction manager e come hai annotato le clsse e del
    database in memoria}
La scelta di usare JPA nella modellazione del dominio ci è sembrata particolarmente vantaggiosa
dal momento che la nostra applicazione non doveva interfacciarsi con un database esistente autonomamente,
ovvero l'app e lo schema del database sarebbero stati sviluppati insieme. 
Le annotazioni JPA nelle classi di dominio
sono servite quindi non soltanto come specifica per l'object-relational mapping, ma grazie alla property
di bootstrap \texttt{hibernate.hbm2ddl.auto = create-drop}, anche per definire univocamente lo schema del database.
In questo modo 
\begin{itemize}
    \item non si pongono problemi di allineamento fra schema db e ORM, in quanto il primo
        è automaticamente desunto dal secondo
    \item la modellazione del dominio può essere rivista e modificata a piacere, incoraggiando
        la ricerca di soluzioni efficaci anche quando scomode da mappare manualmente
\end{itemize}


\todo[color=blue]{Se vuoi scrivi della gui in generale}
\todo[color=green]{Nicco crea gli use case template scegline qualcuno che ritieni significativo NON login e register}