\section{Introduzione generale}

\subsection{Statement}
Il sistema progettato vuole 
//
\begin{itemize}
    \item \textbf{Lista Collegata}
    \item \textbf{ABR}
    \item \textbf{Hash}
\end{itemize}
Per confrontare le varie implementazioni misureremo le prestazioni nell'eseguire le operazioni principali di un dizionario ovvero:
\begin{itemize}
    \item \textbf{Inserimento}
    \item \textbf{Ricerca}
    \item \textbf{Cancellazione}
\end{itemize}
\subsection{Descrizione dello svolgimento dell'esperimento}
Per fare ciò suddivideremo la descrizione dell'esperimento in quattro parti:
\begin{itemize}
    \item \textbf{Spiegazione teorica}: in questa parte descriveremo dal punto di vista teorico il problema considerando tutte e tre le strutture dati
    \item \textbf{Documentazione del codice}: in questa parte forniremo la documentazione del codice 
    e analizzeremo le varie scelte implementative
    \item \textbf{Descrizione delle misurazioni effettuate}: in questa parte spiegheremo le misurazioni effettuate cercando di verificare le ipotesi teoriche
    \item \textbf{Analisi dei risultati sperimentali}: una volta effettuate tutte le varie misurazioni analizzeremo i risultati traendone delle conclusioni
\end{itemize}
\subsection{Specifiche della piattaforma di test}
La piattaforma di test utilizzata per svolgere questo esperimento è la seguente:
\begin{itemize}
    \item \textbf{OS} : Pop!\_OS 22.04 LTS 64-bit
    \item \textbf{CPU} : AMD® Ryzen 5 5500u 2.1 GHz 6 core 12 thread
    \item \textbf{RAM} : Samsung 8GB DDR4 3200MHz
    \item \textbf{SSD} : Western Digital PC SN530 512GB 
\end{itemize}
Il linguaggio di programmazione utilizzato è Python, l'IDE utilizzato per la scrittura e l'esecuzione del codice è \textbf{PyCharm 2024.1.1 (Professional Edition)}. La stesura di questo testo è stata realizzata tramite l'utilizzo dell'editor online \textbf{Overleaf}.