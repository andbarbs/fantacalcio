\subsection{Use Case Templates}

\begin{table}[H]
\caption{Crea fantalega}
\label{UC-01}

\begin{tabularx}{\textwidth}{|l|X|}
\hline
\textbf{Id} & UC-01 (Crea fantalega) \\
\hline
\textbf{Scope} & user goal \\
\hline
\textbf{Descrizione} & L'utente vuole creare una nuova fantalega di cui sarà admin. \\
\hline
\textbf{Attori} & Admin \\
\hline
\textbf{Flusso base} &
\begin{enumerate}[leftmargin=*]
    \item L'utente inserisce il nome e il codice della lega che vuole creare.
    \item Il sistema verifica che non sia già presente una lega con lo stesso codice.
    \item Se i dati sono validi, il sistema crea la nuova lega.
\end{enumerate} \\
\hline
\textbf{Flusso alternativo} &
\begin{enumerate}[leftmargin=*,label=2.\arabic*]
    \item Il codice scelto è già utilizzato per un'altra lega.
    \item Il sistema mostra a schermo un messaggio di errore, specificando la causa del problema.
\end{enumerate} \\
\hline
\textbf{Test} & \hyperref[IT3]{IT3} \\
\hline
\end{tabularx}

\end{table}



\begin{table}[H]
\caption{Assegna giocatori alle rose}
\label{UC-02}

\begin{tabularx}{\textwidth}{|l|X|}
\hline
\textbf{Id} & UC-02 (Assegna giocatori alle rose) \\
\hline
\textbf{Scope} & user goal \\
\hline
\textbf{Descrizione} & L'admin vuole assegnare i giocatori corretti alle rose presenti nella lega. \\
\hline
\textbf{Attori} & Admin \\
\hline
\textbf{Flusso base} &
\begin{enumerate}[leftmargin=*]
    \item L'admin inserisce il team e il giocatore che deve essere assegnato a tale team.
    \item Il sistema verifica che il numero massimo di giocatori nel team non sia già stato raggiunto.
    \item Il sistema verifica che il numero massimo di giocatori nel team appartenenti allo 
            stesso ruolo del nuovo giocatore non sia già stato raggiunto.
    \item Il sistema salva il nuovo contratto tra team e giocatore.
\end{enumerate} \\
\hline
\textbf{Flusso alternativo} &
\begin{enumerate}[leftmargin=*,label=2.\arabic*]
    \item Il numero massimo di giocatori nel team è già stato raggiunto.
    \item Il sistema mostra a schermo un messaggio di errore, specificando la causa del problema.
\end{enumerate}
\begin{enumerate}[leftmargin=*,label=3.\arabic*]
    \item Il numero massimo di giocatori nel team appartenenti allo stesso ruolo
            del nuovo giocatore è già stato raggiunto.
    \item Il sistema mostra a schermo un messaggio di errore, specificando la causa del problema.
\end{enumerate} \\
\hline
\textbf{Test} & \hyperref[IT5]{IT5} \\
\hline
\end{tabularx}

\end{table}



\begin{table}[H]
\caption{Inserisci formazione}
\label{UC-03}

\begin{tabularx}{\textwidth}{|l|X|}
\hline
\textbf{Id} & UC-03 (Inserisci formazione) \\
\hline
\textbf{Scope} & user goal \\
\hline
\textbf{Descrizione} & L'utente vuole inserire la formazione per giocare la partita successiva. \\
\hline
\textbf{Attori} & FantaUser \\
\hline
\textbf{Flusso base} &
\begin{enumerate}[leftmargin=*]
    \item L'utente fornisce la LineUp che vuole utilizzare per la partita successiva.
    \item Il sistema verifica che la data in cui viene effettuata l'operazione 
            sia precedente alla data della partita.
    \item Il sistema verifica che la data in cui viene effettuata l'operazione 
            non sia un sabato o una domenica.
    \item Se la partita non è la prima del campionato, il sistema verifica che 
            i voti per la partita precedente siano già stati calcolati. 
    \item Il sistema verifica che i giocatori appartenenti alla formazione 
            siano giocatori posseduti dall'utente.
    \item Se l'utente ha già stata inserito una formazione per la partita, 
            il sistema la elimina per poter inserire la nuova formazione.
    \item Il sistema salva la formazione per la partita successiva.
\end{enumerate} \\
\hline
\textbf{Flusso alternativo} &
\begin{enumerate}[leftmargin=*,label=2.\arabic*]
    \item La data in cui viene effettuata l'operazione è successiva alla data della partita.
    \item Il sistema mostra a schermo un messaggio di errore, specificando la causa del problema.
\end{enumerate}
\begin{enumerate}[leftmargin=*,label=3.\arabic*]
    \item La data in cui viene effettuata l'operazione è un sabato o una domenica.
    \item Il sistema mostra a schermo un messaggio di errore, specificando la causa del problema.
\end{enumerate} 
\begin{enumerate}[leftmargin=*,label=4.\arabic*]
    \item I voti per la partita precedente non sono stati ancora calcolati.
    \item Il sistema mostra a schermo un messaggio di errore, specificando la causa del problema.
\end{enumerate}
\begin{enumerate}[leftmargin=*,label=5.\arabic*]
    \item C'è almeno un giocatore all'interno della formazione che non appartiene all'utente.
    \item Il sistema mostra a schermo un messaggio di errore, specificando la causa del problema.
\end{enumerate} \\
\hline
\textbf{Test} & \hyperref[IT9]{IT9} \\
\hline
\end{tabularx}

\end{table}



\begin{table}[H]
\caption{Scambia giocatori - Invia proposta}
\label{UC-04}

\begin{tabularx}{\textwidth}{|l|X|}
\hline
\textbf{Id} & UC-04 (Scambia giocatori - Invia proposta) \\
\hline
\textbf{Scope} & user goal \\
\hline
\textbf{Descrizione} & L'utente vuole inviare una proposta di scambio di giocatori ad un altro utente. \\
\hline
\textbf{Attori} & FantaUser \\
\hline
\textbf{Flusso base} &
\begin{enumerate}[leftmargin=*]
    \item L'utente fornisce i giocatori coinvolti nello scambio e le relative rose.
    \item Il sistema verifica che i due giocatori abbiano lo stesso ruolo.
    \item Il sistema verifica che i giocatori coinvolti appartengano alle rose fornite dall'utente.
    \item Viene creata la nuova proposta e si verifica se è già presente una 
            proposta con le stesse caratteristiche.
    \item Il sistema salva la nuova proposta.
\end{enumerate} \\
\hline
\textbf{Flusso alternativo} &
\begin{enumerate}[leftmargin=*,label=2.\arabic*]
    \item I due giocatori coinvolti nello scambio non hanno lo stesso ruolo.
    \item Il sistema mostra a schermo un messaggio di errore, specificando la causa del problema.
\end{enumerate}
\begin{enumerate}[leftmargin=*,label=3.\arabic*]
    \item I giocatori coinvolti nello scambio non appartengono alle rose fornite dall'utente
    \item Il sistema mostra a schermo un messaggio di errore, specificando la causa del problema.
\end{enumerate} 
\begin{enumerate}[leftmargin=*,label=4.\arabic*]
    \item Nel sistema è già presente una proposta con le stesse caratteristiche.
    \item Il sistema mostra a schermo un messaggio di errore, specificando la causa del problema.
\end{enumerate} \\
\hline
\textbf{Test} & \hyperref[IT12]{IT12} \\
\hline
\end{tabularx}

\end{table}



\begin{table}[H]
\caption{Scambia giocatori - Accetta proposta}
\label{UC-05}

\begin{tabularx}{\textwidth}{|l|X|}
\hline
\textbf{Id} & UC-05 (Scambia giocatori - Accetta proposta) \\
\hline
\textbf{Scope} & user goal \\
\hline
\textbf{Descrizione} & L'utente vuole accettare una proposta di scambio di giocatori inviata da un altro utente. \\
\hline
\textbf{Attori} & FantaUser \\
\hline
\textbf{Flusso base} &
\begin{enumerate}[leftmargin=*]
    \item L'utente fornisce la proposta che vuole accettare e la propria rosa.
    \item Il sistema verifica che la rosa fornita sia coinvolta nella proposta.
    \item Il sistema verifica che entrambi i contratti dei giocatori coinvolti siano ancora presenti.
    \item La proposta è accettata, quindi il sistema provvede a modificare i contratti dei giocatori
            per cambiare la loro squadra di appartenenza.
    \item Il sistema salva i nuovi contratti. 
\end{enumerate} \\
\hline
\textbf{Flusso alternativo} &
\begin{enumerate}[leftmargin=*,label=2.\arabic*]
    \item La rosa fornita non è coinvolta nella proposta.
    \item Il sistema mostra a schermo un messaggio di errore, specificando la causa del problema.
\end{enumerate}
\begin{enumerate}[leftmargin=*,label=3.\arabic*]
    \item Il contratto di almeno uno dei giocatori coinvolti non è più presente.
    \item Il sistema provvede a rifiutare la proposta dato che non è valida.
    \item Il sistema mostra a schermo un messaggio di errore, specificando la causa del problema.
\end{enumerate} \\
\hline
\textbf{Test} & \hyperref[IT13]{IT13} \\
\hline
\end{tabularx}

\end{table}