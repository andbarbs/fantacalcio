\documentclass{article}
%% Useful packages
\usepackage[utf8]{inputenc}
\usepackage[a4paper,left=3.5cm,right=3.5cm,top=2cm,bottom=2cm]{geometry}
\usepackage{crop,graphicx,amsmath,array,color,amssymb,fancyhdr,lineno}
\usepackage{flushend,stfloats,amsthm,chngpage,times,,lipsum,lastpage} 
\usepackage{calc,listings,color,wrapfig,tabularx,longtable,enumitem}
\usepackage{multirow}
\usepackage{caption}
\usepackage{subcaption}
\usepackage{xcolor}
\usepackage{tcolorbox}
\definecolor{shadecolor}{rgb}{0.86,0.86,0.86}
\usepackage{float}
\usepackage{tikz}
\usetikzlibrary{calc,shapes.multipart,chains,arrows}


\usepackage{lineno}
\usepackage{csquotes}
\usepackage[italian]{babel}
%%%%%%%%%%%%   Header and Footer  %%%%%%%%%%%%%

\pagestyle{fancy}
\fancypagestyle{plain}{%
  \renewcommand{\headrulewidth}{0pt}%
  \fancyhf{}%
}

\title{%
    Architettura ed Implementazione di un sistema per il gioco del Fantacalcio}


\begin{document}


\fancyhf{}
\fancyhead[R]{Ingegneria del Software}
\fancyfoot[R]{ \large \bf \thepage\ \centering}%

\newpage
\tableofcontents
\listoffigures

\newpage
\include{Part/Introduzione Generale}
\newpage
\include{Part/Confronto implementazione dizionari}
\begin{thebibliography}{9}
\bibitem{algLib}
\todo{Cosa ci mettiamo libro del vicario su jpa? poi? agag}

\end{thebibliography}


\end{document}